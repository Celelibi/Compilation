\documentclass[a4paper]{article}

\usepackage[T1]{fontenc}
\usepackage[francais]{babel}
\usepackage{fullpage}
\usepackage[vlined,ruled]{algorithm2e}
\usepackage{amssymb}
\usepackage{amsmath,alltt}

\newenvironment{algo}[1]{%
\vspace{1cm}
\renewcommand{\algorithmcfname}{Algorithm}
\begin{algorithm}[H]
\label{#1}
\caption{#1}
\SetKwInOut{Constant}{Constant}
\SetKwInOut{Input}{Input}
\SetKwInOut{Output}{Output}
\SetKwInOut{Global}{Global}
\SetKwInOut{Local}{Local}
}{%
\end{algorithm}
\vspace{1cm}
}


\newcommand{\true}{\mbox{\it true}}
\newcommand{\false}{\mbox{\it false}}
\newcommand{\Boolean}{\{\true,\false\}}
\newcommand{\Integer}{\mathbb{Z}}
\newcommand{\Complex}{\mathbb{C}}
\newcommand{\Real}{\mathbb{R}}

\begin{document}
\title{Projet de compilation}
\maketitle

Seul les algorithmes sont pris en compte pour la compilation. Tout le
reste du document \LaTeX \; est ignor\'e. Le point d'entr\'ee est
l'algorithme \textbf{main}.

\begin{algo}{main}
\Input{$\emptyset$}
\Output{$\emptyset$}
\Global{$\emptyset$}
\Local{$accu \in \Integer$}
\BlankLine
\If{$(accu < 5 \vee accu > 10) \wedge (5 + -4000 \div 100 > 500)$}{$accu \leftarrow 5$\;}
%$accu \leftarrow 1 + 5 \times -6$ \;
%$accu \leftarrow accu \times -12$ \;
\end{algo}



\end{document}
